\documentclass[a5paper,12pt]{scrartcl}
\usepackage[utf8]{inputenc}
\usepackage[lf]{electrum}
\usepackage[T1]{fontenc}
\usepackage[hidelinks,backref,hyperfootnotes=false,plainpages=false]{hyperref}
\usepackage{microtype}
\usepackage[ngerman]{babel}
\addtokomafont{disposition}{\rmfamily} % Sections auch in Serifenschrift

\usepackage{tweaklist}
\renewcommand{\itemhook}{\setlength{\leftmargin}{1.8em}\setlength{\topsep}{4pt}\setlength{\itemsep}{0pt}}
\renewcommand{\enumhook}{\setlength{\leftmargin}{1.8em}}

\begin{document}

\subject{}
\title{\raggedright{}Dein Gerät,\newline{}Deine Freiheit,\newline{}Deine Wahl}
\subtitle{\raggedright{}Wie Du mit freier Software zu einer\\freien Gesellschaft beitragen kannst}
\author{}
\date{} % damit kein Datum gesetzt wird
\maketitle

\noindent{}von Sven Hartenstein

\vspace{1em}

\noindent{}Version 2021-01-16

\thispagestyle{empty}

\pagebreak
\tableofcontents
\pagebreak


\section{Unser Ziel: Freiheit}

Immer wieder sind auf unserem Planeten Freiheiten von Menschen bedroht
und immer wieder wird für sie gestritten: die Freiheit, die eigene
Meinung zu sagen, die Freiheit, Kritisches zu schreiben und zu
veröffentlichen, die Freiheit, sich so zu kleiden wie man möchte, die
Freiheit, bei politischen Wahlen die eigenen Interessen zu vertreten,
die Freiheit von Wissenschaftler*innen, Ergebnisse ihrer Forschung
unabhängig zu veröffentlichen. Das sind nur einige Beispiele.

Freiheit bedeutet, dass wir angstfrei \textbf{so leben können, wie wir
  leben möchten}.

Die Freiheiten, um die es in diesem Text geht, sind unsere Freiheiten
als Nutzer*innen von Computern. Können wir unsere Geräte so nutzen,
wie wir möchten? Können wir kontrollieren, dass Geräte nur das tun,
was wir möchten? Können wir selbstständig verbessern, wie Computer für
uns Aufgaben erledigen? Du erfährst, was freie Software ist, wie sie
zu unserer Freiheit beiträgt und wie Nutzer*innen von Computern
gemeinsam für ihre Freiheit kämpfen.


\section{Ein paar Begriffe erklärt}

Das Wort \textbf{Software} bezeichnet Programme, die auf Computern
laufen und von diesen ausgeführt werden. Zu Software gehören das
Betriebssystem (z.\,B.\ GNU/Linux, Android oder Microsoft Windows) und
Anwendungen (z.\,B.\ Office, Internetbrowser wie Firefox oder
Medienspieler wie VLC).

Software wird \textbf{programmiert}. Das heißt, dass ein
\textbf{Quellcode}, ein für Menschen lesbarer Text, in einer
bestimmten Programmiersprache geschrieben wird. Dieser Quellcode wird
vom Computer meist in ein maschinen-lesbares Programm übersetzt, damit
es vom Betriebssystem ausgeführt werden kann. Nur aus dem Quellcode
kann man ablesen, was die Software genau tut.

\textbf{Computer} sind Geräte, die mittels programmierter Software
arbeiten. Dazu zählen beispielsweise Desktop-Computer, Laptops,
Smartphones und Server\footnote{Ein Server ist ein Computer, von dem
  andere Computer Informationen (z.\,B.\ Websites) oder andere Daten
  abrufen können.}. Aber auch in unzähligen Maschinen wie Autos,
medizinischen Geräten, Drohnen oder Geldautomaten stecken Computer,
auf denen Software läuft.


\section{Freie Software: vier Freiheiten}

Freie Software achtet die \textbf{Freiheit der Nutzer*innen}, also der
Menschen, die Computer verwenden. Wir nennen eine bestimmte Software
freie Software, wenn sie vier bestimmte Freiheiten garantiert:
\begin{enumerate}
\item Die Software darf \textbf{zu jedem Zweck genutzt} werden. Das
  heißt: Die Nutzer*innen entscheiden, was sie mit einem Programm
  machen, und nicht die Entwickler*innen.
\item Die Software erlaubt den Nutzer*innen, die Funktionsweise des
  Programms zu \textbf{untersuchen} und nach ihren Bedürfnissen zu
  \textbf{verändern}. Dazu muss der Quellcode der Software verfügbar gemacht werden.
\end{enumerate}

Die meisten Menschen, die Computer nutzen, programmieren nicht
selbst. Sie können den Quelltext nicht verstehen und nicht
verändern. Darum sind zwei weitere Freiheiten notwendig, damit die
Nutzer*innen gemeinsam (kollektiv) frei sind, Software nach ihren
Bedürfnissen weiterzuentwickeln.

\begin{enumerate}
\item[3.] Die Software darf \textbf{weiterverbreitet}
  werden. Nutzer*innen dürfen Kopien der Software weitergeben und
  damit andere Menschen unterstützen.
\item[4.] Die Software darf auch \textbf{verändert weitergegeben}
  werden. Dadurch sind Nutzer*innen bei der Entwicklung der Software
  nicht mehr von einer bestimmten Firma oder Organisation abhängig:
  Sie können selbstständig eine veränderte Abspaltung der Software
  weiterentwickeln oder von jemand anderes weiterentwickeln
  lassen. Auch für diese Freiheit muss der Quellcode verfügbar sein.
\end{enumerate}

Software, die diese vier Freiheiten nicht gewährt, nennen wir
\textbf{proprietäre Software}. Proprietär bedeutet "`in Eigentum
befindlich"'.


\section{Vorteile für Nutzer*innen}

Wenn Nutzer*innen gemeinsam Software nach ihren Wünschen entwickeln
können, haben sie mehr \textbf{Kontrolle} darüber und können sie
\textbf{selbst entscheiden}, was die Software auf ihren Geräten
tut. Die Software kann dann so programmiert werden, dass sie den
\textbf{Bedürfnissen der Nutzer*innen} dient anstatt den Interessen
von Firmen. Außerdem führt freie Software dazu, dass Nutzer*innen sich
stärker gegenseitig \textbf{unterstützen und zusammenarbeiten}
(kooperieren) können, weil sie die Software miteinander teilen und
füreinander anpassen können.

Bei proprietärer Software ist das häufig nicht so: Die Firma, die
Programmierer*innen für ihre Arbeit bezahlt, entscheidet, was die
Software tut und was davon die Nutzer*innen erfahren. Nutzer*innen
können die Software nicht so verändern, wie sie möchten. Durch die
Software hat die Firma Macht über die Nutzer*innen.

Weil der Quellcode freier Software von vielen Menschen gelesen werden
kann, können Fehler leichter gefunden und behoben werden. Bekannte
freie Software gilt daher als \textbf{sicherer} als proprietäre
Software. Das bedeutet, dass sie weniger Schwachstellen enthält, durch
die sie von außen manipuliert werden kann.


\section{Negative Merkmale proprietärer Software}

Wenn Nutzer*innen keine Kontrolle über die von ihnen genutzte Software
haben, kann es passieren, dass Nutzer*innen und Entwickler*innen nicht
zusammenarbeiten, sondern Interessen der Nutzer*innen verletzt
werden. Das kommt häufig vor. Hier ein paar Beispiele:

\begin{itemize}
\item Nutzer*innen werden ausgespäht.
\item Firmen sammeln Informationen über die Nutzer*innen, z.\,B.\ ihre
  Interessen oder ihren Standort.
\item Nutzer*innen werden daran gehindert, bestimmte Dinge zu tun,
  z.\,B.\ DVDs auszulesen.
\item Nutzer*innen werden daran gehindert, anderen Menschen eine Kopie
  der Software zu geben und ihnen dadurch zu helfen.
\item Nutzer*innen werden daran gehindert, die Software so zu
  verändern, dass sie ihnen besser gefällt, und diese Veränderung
  weiterzugeben.
\item Firmen bauen in Software eine Hintertür ein, mit der sie zu
  jeder Zeit Dateien auslesen oder im Hintergrund Programme auf dem
  Computer der Nutzer*innen ausführen können.
\item Funktionen (z.\,B.\ das Lesen von Büchern oder das Abspielen von
  Musik) werden den Nutzer*innen nur für eine bestimmte Zeit
  ermöglicht, damit sie regelmäßig Geld für eine Verlängerung zahlen.
\item Nutzer*innen werden unter Druck gesetzt: Entweder sie
  akzeptieren die Nutzungsbedingungen der Firmen oder sie können die
  Software nicht nutzen.
\item Nutzer*innen können Software nur mit Geräten eines bestimmten
  Herstellers nutzen oder Geräte nur mit einer bestimmten Software.
\end{itemize}

Selbst wenn bestimmte Softwarefirmen all das nicht tun: Sie sollten
gar nicht die Möglichkeit dazu haben. Solange Software proprietär ist,
können wir nicht sicher sein, was sie auf unseren Geräten
tut. Vielleicht vertrauen wir einer bestimmten Firma und verlassen uns
auf ihre Software -- aber was, wenn diese Firma verkauft wird?

In der Regel bauen Firmen schädliche Funktionen nicht aus böser
Absicht in Software ein, sondern weil sie erwarten, dadurch mehr Geld
zu verdienen.


\section{Die freie Welt gibt es schon (in Sachen Software)}

Durch sehr viel Arbeit von Programmierer*innen, die freie Software
geschrieben haben, können Nutzer*innen heute die meisten Aufgaben mit
freier Software erledigen: Dokumente schreiben, im Internet surfen,
Musik komponieren, Server betreiben, Berechnungen durchführen, Spiele
spielen und vieles, vieles mehr. Es gibt tausende Programme, die den
Nutzer*innen die vier Freiheiten garantieren.

Hier ein paar bekannte Beispiele für freie Software:
\begin{itemize}
\item die Betriebssysteme GNU/Linux und Android\footnote{Der Großteil
    von Android, das vor allem auf Smartphones und Tablets eingesetzt
    wird, ist freie Software. Auf den meisten Geräten wird von den
    Herstellern aber auch proprietäre Software installiert. Viele
    bekannte Apps (z.\,B.\ von Google) sind proprietär. F-Droid ist
    ein App-Store für freie Software.}
\item die Office-Pakete LibreOffice und OpenOffice
\item der Internetbrowser Firefox und das E-Mail-Programm Thunderbird
\item der Medienspieler VLC
\item die Bildbearbeitungssoftware GIMP und die Zeichensoftware
  Inkscape
\item die Messenger Signal, Telegram und Element
\end{itemize}

Neben diesen Beispielen, die viele Menschen bereits kennen, gibt es
hochwertige freie Software auch für spezifische Aufgaben, wie zum
Beispiel einen Server betreiben, Daten statistisch auswerten,
Unterschiede zwischen Dateien anzeigen oder Patient*innen einer
Arztpraxis verwalten.

Das Betriebssystem \textbf{GNU/Linux} wird von verschiedenen
Anbieter*innen in verschiedenen Varianten und mit unterschiedlicher
Auswahl von Anwendungen zur Verfügung gestellt. Diese
Zusammenstellungen nennen wir
GNU/Linux-\textbf{Distributionen}. Beispiele sind Debian, Ubuntu,
Fedora oder PureOS. Einige Distributionen enthalten nur freie
Software, andere auch proprietäre Software.

Das Angebot an freier Software ist so groß und die Möglichkeiten, sie
anzupassen, sind so vielfältig, dass die Anwendung -- insbesondere zu
Beginn -- anstrengend sein kann: Welche Distribution möchte ich
installieren? Welche graphische Oberfläche gefällt mir am besten?
Welchen Browser, Dateimanager und Bildbetrachter wähle ich? Unzählige
\textbf{Entscheidungen} sind zu treffen und das ist nicht immer
einfach. Der Austausch mit Freund*innen und Empfehlungen im Internet
können bei der "`Qual der Wahl"' helfen. Und: Vielen Menschen macht
die Beschäftigung mit freier Software sogar großen \textbf{Spaß}, weil
es viel zu entdecken und auszutauschen gibt, und weil viele freie
Anwendungen einfach super sind.


\section{Lizenzen garantieren Freiheiten (oder Unfreiheiten)}

Nach den Gesetzen (dem Urheberrecht) in vielen Ländern gehören Werke
den Personen, die sie geschaffen haben -- das gilt auch für
Software. Wenn wir proprietäre Software weitergeben oder verändern
würden, können wir gezwungen werden, damit aufzuhören oder sogar
bestraft werden.

Woher wissen wir also, welche Software uns die vier Freiheiten
gewährt? Dafür gibt es \textbf{Lizenzen}. Das sind schriftliche
Genehmigungen, die Programmierer*innen mit der Software weitergeben,
und in denen die vier Freiheiten ausdrücklich gewährt
werden. Beispiele für häufig von Programmierer*innen freier Software
genutzte Lizenzen sind die \textbf{GNU General Public License} (GPL)
oder die \textbf{Apache Lizenz}.


\section{"`Freie"'\,oder\,"`Open\,Source"'-Software?}

Häufig wird Software, deren Quelltext verfügbar ist, "`\textbf{Open
  Source}"'-Software genannt. Meist geht es dabei um die gleiche
Software, die andere "`\textbf{freie Software}"' nennen.

Ist es egal, welchen Begriff wir verwenden? Nicht ganz! Die Begriffen
stellen verschiedene \textbf{Ziele} in den Vordergrund. "`Open
Source"' hebt vor allem die Art der Entwicklung von Software hervor,
die zu \textbf{leistungsstarker Software} beitragen kann: offener
Quellcode. "`Freie Software"' betont hingegen das Ziel, dass Software
uns Menschen \textbf{Freiheiten garantieren} soll. Darum ist es
hilfreich, von "`freier Software"' zu sprechen: Damit wird deutlich,
dass wir einen ethischen oder politischen Anspruch haben und die
Menschen, die uns zuhören, verstehen leichter, worum es uns geht.

Übrigens: Der Begriff "`\textbf{Freeware}"' meint etwas ganz anderes
als "`freie Software"', auch wenn er so ähnlich klingt. "`Freeware"'
meint kostenlose proprietäre Software, die nicht die vier Freiheiten
gewährt. Das englische Wort "`free"' bedeutet sowohl "`frei"' als auch
"`kostenlos"'. Freeware ist kostenlos; freie Software wird auch häufig
kostenlos angeboten, garantiert aber die vier Freiheiten.


\section{Wo freie Software besonders wichtig ist}

Menschen sollten frei sein. Darum ist es besonders wichtig, dass
Menschen auf ihren \textbf{privaten Geräten} freie Software
einsetzen. Besonders bedeutsam kann das für politische
\textbf{Aktivist*innen} oder \textbf{Journalist*innen} sein, damit sie
unabhängig und vertraulich arbeiten können.

Die Aufgabe von \textbf{Schulen und Hochschulen} ist, Kindern und
Erwachsenen beizubringen, zusammenzuarbeiten, und sie dabei zu
unterstützen, selbstständig Aufgaben erledigen zu können -- unabhängig
von bestimmten Produkten bestimmter Firmen. An Schulen und Hochschulen
soll Wissen miteinander geteilt werden -- auch das Wissen darüber, wie
Software arbeitet. All das ist nur mit freier Software zu erreichen.

In \textbf{Forschung und Wissenschaft} ist entscheidend, dass
Wissenschaftler*innen unabhängig arbeiten können, und dass sie
verstehen und kontrollieren können, was ihre Computer tun. Das ist
nicht nur für die Forschenden selbst wichtig, sondern für die ganze
Gesellschaft, die sich auf Forschungsergebnisse verlässt.

Ähnlich ist es mit \textbf{Behörden und Regierungen}. Diese sollen der
Bevölkerung dienen und dürfen dabei nicht von proprietärer Software
abhängig sein. Hier ist die Sicherheit besonders wichtig: Behörden und
Regierungen müssen die Kontrolle darüber haben, dass ihre Software
nicht im Hintergrund Daten an Dritte weiterleitet oder gehackt werden
kann.

In \textbf{Arztpraxen und Kliniken} sollten Me"-di"-zi"-ner*in"-nen
die Kontrolle über ihre Geräte haben. Auch Patient*innen sollten
verstehen und bestimmen können, welche Software beispielsweise ihren
Herzschrittmacher oder ihre Prothese steuert.

In der \textbf{Landwirtschaft} kann freie Software dazu beitragen,
dass Landwirt*innen nicht von großen Firmen abhängig sind, sondern
ihre Geräte selbst steuern und leichter reparieren können. Die
Unabhängigkeit von Landwirt*innen ist wichtig, damit sie sich an den
Wünschen der Verbraucher*innen orientieren können.


\section{Was tun? Dein Engagement für freie Software}

Menschen können auf viele verschiedene Weisen zur Weiterentwicklung
freier Software und damit zur Freiheit der Nutzer*innen
\textbf{beitragen}. Manche Wege erfordern bestimmte Kenntnisse oder
Fähigkeiten, andere kann fast jede*r gehen. Du kannst

\begin{itemize}
\item freie Software \textbf{verwenden} und damit Deine eigene
  Freiheit sicherstellen,
\item \textbf{andere unterstützen}, freie Software zu nutzen oder zu
  installieren, und aufhören, anderen Menschen Software zu empfehlen,
  die proprietär ist,
\item \textbf{andere bitten}, Dich beim Installieren oder Nutzen
  freier Software zu unterstützen,
\item Dich selbst und andere \textbf{informieren}, um bei der Auswahl
  von Software gute Entscheidungen treffen zu können,
\item selbst freie Software \textbf{programmieren} oder als
  Programmierer*in weiterentwickeln,
\item Programmierer*innen \textbf{mitteilen}, wie sie die Software für
  Dich verbessern können, oder welche Fehler Du in ihrer Software
  gefunden hast,
\item zur \textbf{Dokumentation} freier Software beitragen (z.\,B.\
  Handbuch, Anleitungen),
\item zum \textbf{Design} freier Software beitragen (z.\,B.\ benötigte
  Symbole zeichnen),
\item Software in andere Sprachen \textbf{übersetzen},
\item Deiner Schule oder Hochschule, Deinem Arbeitgeber, Deinem Verein
  und Deinen Freunden \textbf{sagen, dass Du mit freier Software
  arbeiten möchtest},
\item \textbf{"`freie Software"' sagen} (statt
  "`Open-Source-Software"'),
\item \textbf{Nein sagen}, wenn Du aufgefordert wirst, proprietäre
  Software zu nutzen,
\item \textbf{auf etwas Bequemlichkeit verzichten}, wenn die Nutzung
  proprietärer Software für eine bestimmte Aufgabe einfacher ist als
  die Nutzung freier Software,
\item für freie Software \textbf{Geld bezahlen oder spenden} an
  Menschen oder Organisationen, die freie Software programmieren oder
  zur Verfügung stellen, und
\item mit anderen \textbf{Kampagnen oder Veranstaltungen} zum Thema
  freie Software organisieren.
\end{itemize}

Anfangs kann es schwierig sein, freie Software für eine bestimmte
Aufgabe zu finden, zu installieren oder zu nutzen. Davon solltest Du
Dich nicht entmutigen lassen: Jedes noch so kleine Engagement für
freie Software hilft und jeder noch so kleine \textbf{Schritt in die
  richtige Richtung} macht unsere Welt ein bisschen besser.

Im Internet findest Du ganz viele Informationen über freie Software
allgemein und über bestimmte freie Anwendungen. Organisationen, die
sich für freie Software engagieren und bei denen Du Dich informieren
und engagieren kannst, sind beispielweise die Free Software Foundation
Europe\footnote{https://fsfe.org}, die Free Software
Foundation\footnote{https://www.fsf.org} aus den USA, das
GNU-Projekt\footnote{https://www.gnu.org} und
Digitalcourage\footnote{https://digitalcourage.de}.


\section{Blick über den Tellerrand}

In dieser Einführung geht es um Software und ihre Rolle für unsere
Freiheit bei der Nutzung von Geräten, auf denen Software läuft. Diese
Freiheit hängt allerdings von mehr als der eingesetzten Software
ab. In diesem Abschnitt wird darum kurz auf verwandte Themen
hingewiesen.

Auch \textbf{Hardware} sollte frei sein. Das bedeutet, dass die
Baupläne für Geräte und Einzelteile (z.\,B.\ Chips, Grafikkarten oder
Drucker) frei sein sollten. Wenn Hersteller von Geräten oder
Einzelteilen offenlegen, wie ihre Produkte von Software gesteuert
werden können, ist es viel einfacher, freie Software für die Geräte zu
programmieren. Außerdem ist dann nachvollziehbar, was die Geräte tun
und ob sie sicher sind.

Wenn wir online sind, senden wir bewusst oder unbewusst Daten an die
Server von sozialen Netzwerken, Online-Shops oder anderen
Anbietern. Hier ist \textbf{informationelle Selbstbestimmung} ein
wichtiges Thema. Viele Menschen kämpfen beispielsweise für
Datenschutzgesetze, die Nutzer*innen davor schützen, dass ihre Daten
missbraucht werden.

Einige soziale Netzwerke werden zentral von Firmen betrieben. Wenn wir
dort Texte, Fotos, Videos oder anderes teilen, werden sie auf den
Servern der Firma gespeichert und verarbeitet. Einige Netzwerke haben
fast eine Monopolstellung und viel Macht über ihre Nutzer*innen. Darum
haben Menschen alternative \textbf{soziale Netzwerke} geschaffen, die
den Nutzer*innen mehr Kontrolle über ihre Daten bieten, die verteilt
auf Servern verschiedener Menschen oder Organisationen laufen und für
die freie Software zur Verfügung steht -- sowohl für die Nutzer*innen
als auch auf dem Server. Ein paar bekanntere Beispiele sind die
Microblogging-Software Mastodon\footnote{https://joinmastodon.org},
die Videoplattform Peertube\footnote{https://joinpeertube.org/de/} und
das Kommunikationsprotokoll Matrix\footnote{https://matrix.org}.

Lizenzen, die Freiheiten gewähren, gibt es nicht nur für Software,
sondern auch für verschiedenste \textbf{andere Werke}, die dem
Urheberrecht unterliegen, z.\,B.\ Bücher, Musik,
Graphiken.\footnote{Informationen darüber findest Du unter
  https://creativecommons.org .}


\section{Schluss}

Freie Meinungsäußerung, Bewegungsfreiheit, Versammlungsfreiheit, freie
Presse, freie Wahlen und einige Freiheiten mehr gelten heute in vielen
Ländern als Menschenrechte, die jedem Menschen zustehen. Die Erfahrung
zeigt, dass der \textbf{Kampf um Freiheiten} viel Zeit und Energie
verlangt: Immer wieder werden Freiheiten für bestimmte Personen, an
bestimmten Orten oder in bestimmten Situationen eingeschränkt. Darum
setzen sich immer wieder Menschen für Freiheit ein.

Computer spielen heute eine riesige Rolle in den meisten
Lebensbereichen und Software hat großen Einfluss auf unser Leben --
wenn sie auf unseren eigenen Geräten läuft, wenn sie auf den Geräten
unserer Schule oder unserer Ärzt*innen oder den Maschinen unserer
Landwirt*innen läuft. Wir sollten als Nutzer*innen und Bürger*innen
darauf achten, dass wir Software gemeinsam kontrollieren, statt von
proprietärer Software und ihren Hersteller*innen abhängig zu sein.


\section{Dieses Dokument ist frei}

Ich, der Autor, habe dieses Werk in die \textbf{Gemeinfreiheit} --
auch Public Domain genannt -- entlassen, indem ich weltweit auf alle
urheberrechtlichen und verwandten Schutzrechte verzichtet habe, soweit
das gesetzlich möglich ist.

\textbf{Du darfst} dieses Werk kopieren, verändern, verbreiten und
aufführen, auch zu kommerziellen Zwecken, ohne um weitere Erlaubnis
bitten zu müssen.

Das Werk "`Dein Gerät, Deine Freiheit, Deine Wahl"' %einschließlich
% Text und Bildern
ist gekennzeichnet mit CC0 1.0 Universell (CCO
1.0). Um eine Kopie dieser Lizenz zu sehen,
besuche\\
https://creativecommons.org/publicdomain/zero/1.0

Ich \textbf{danke} der Free Software Foundation für ihre Arbeit. Fast
alle Gedanken, die diesem Werk zugrunde liegen, habe ich von ihr
gelernt.

Wenn Du \textbf{Verbesserungsvorschläge} für dieses Dokument hast,
möchte ich gerne davon erfahren! Wie Du mit mir \textbf{Kontakt}
aufnehmen und dieses Werk herunterladen kannst, erfährst Du auf
https://svenhartenstein.de\,.

\end{document}

